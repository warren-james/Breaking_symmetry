\PassOptionsToPackage{unicode=true}{hyperref} % options for packages loaded elsewhere
\PassOptionsToPackage{hyphens}{url}
%
\documentclass[
]{article}
\usepackage{lmodern}
\usepackage{amssymb,amsmath}
\usepackage{ifxetex,ifluatex}
\ifnum 0\ifxetex 1\fi\ifluatex 1\fi=0 % if pdftex
  \usepackage[T1]{fontenc}
  \usepackage[utf8]{inputenc}
  \usepackage{textcomp} % provides euro and other symbols
\else % if luatex or xelatex
  \usepackage{unicode-math}
  \defaultfontfeatures{Scale=MatchLowercase}
  \defaultfontfeatures[\rmfamily]{Ligatures=TeX,Scale=1}
\fi
% use upquote if available, for straight quotes in verbatim environments
\IfFileExists{upquote.sty}{\usepackage{upquote}}{}
\IfFileExists{microtype.sty}{% use microtype if available
  \usepackage[]{microtype}
  \UseMicrotypeSet[protrusion]{basicmath} % disable protrusion for tt fonts
}{}
\makeatletter
\@ifundefined{KOMAClassName}{% if non-KOMA class
  \IfFileExists{parskip.sty}{%
    \usepackage{parskip}
  }{% else
    \setlength{\parindent}{0pt}
    \setlength{\parskip}{6pt plus 2pt minus 1pt}}
}{% if KOMA class
  \KOMAoptions{parskip=half}}
\makeatother
\usepackage{xcolor}
\IfFileExists{xurl.sty}{\usepackage{xurl}}{} % add URL line breaks if available
\IfFileExists{bookmark.sty}{\usepackage{bookmark}}{\usepackage{hyperref}}
\hypersetup{
  pdftitle={Supplementary Material for Asymmetry paper},
  pdfborder={0 0 0},
  breaklinks=true}
\urlstyle{same}  % don't use monospace font for urls
\usepackage[margin=1in]{geometry}
\usepackage{longtable,booktabs}
% Allow footnotes in longtable head/foot
\IfFileExists{footnotehyper.sty}{\usepackage{footnotehyper}}{\usepackage{footnote}}
\makesavenoteenv{longtable}
\usepackage{graphicx,grffile}
\makeatletter
\def\maxwidth{\ifdim\Gin@nat@width>\linewidth\linewidth\else\Gin@nat@width\fi}
\def\maxheight{\ifdim\Gin@nat@height>\textheight\textheight\else\Gin@nat@height\fi}
\makeatother
% Scale images if necessary, so that they will not overflow the page
% margins by default, and it is still possible to overwrite the defaults
% using explicit options in \includegraphics[width, height, ...]{}
\setkeys{Gin}{width=\maxwidth,height=\maxheight,keepaspectratio}
\setlength{\emergencystretch}{3em}  % prevent overfull lines
\providecommand{\tightlist}{%
  \setlength{\itemsep}{0pt}\setlength{\parskip}{0pt}}
\setcounter{secnumdepth}{5}
% Redefines (sub)paragraphs to behave more like sections
\ifx\paragraph\undefined\else
  \let\oldparagraph\paragraph
  \renewcommand{\paragraph}[1]{\oldparagraph{#1}\mbox{}}
\fi
\ifx\subparagraph\undefined\else
  \let\oldsubparagraph\subparagraph
  \renewcommand{\subparagraph}[1]{\oldsubparagraph{#1}\mbox{}}
\fi

% set default figure placement to htbp
\makeatletter
\def\fps@figure{htbp}
\makeatother


\title{Supplementary Material for Asymmetry paper}
\author{}
\date{\vspace{-2.5em}}

\begin{document}
\maketitle

{
\setcounter{tocdepth}{2}
\tableofcontents
}
\hypertarget{resampling}{%
\section{Resampling}\label{resampling}}

For experiments 1, 3, and 4 we made use of resampling existing data as we had run experiments in the past that were comparable. Experiment 2 was slightly different in that we did not have previous data to compare to, as such, we generated a series of distributions based on previous data. Resampling was carried out in order to establish if adding more pariticipants to our sample sizes would have lead to a reduction in uncertainty around our estimates.

\hypertarget{probability}{%
\subsection{Probability}\label{probability}}

For this experiment, data from (\(N\) = 12) Clark and Hunt (2016), which acted as a substitute for the Symmetric condition, and an unpublished pilot study (\(N\) = 11), which followed the same rules as the Bias condition in the main experiment, were resampled in order to invesigate how often participants would fixate one of the side boxes given it was more likely to contain the target for the Bias condition. For the Symmetric condition, we followed the same rule as the in the paper and classified the most likely box as the side box that participants fixated most often. These data are shown in the box plots of the figure below. The main interest for our purposes was the proportion of time participants fixated the most likely box.

The data were coded so that a fixation was classified as either being to the Most likely box or not. We then resampled (with replacement) these data by selecting a random \(N\) participants from each condition (ranging from 2 to 20) then sampling 300 trials from these participants from the different datasets. This was done 5000 times to estimate the expected difference between the Symmetric and Bias conditions in terms of proportion of fixations to the Most likely side and the associated certainty around these values.

As can be seen in the figure below, the uncertainty surrounding the estimate for the difference between the groups appears to plateau somewhere around 15 participants. The shaded region represents a 95\% Highest Density Interval (HDI) for the disribution of differences simulated through resampling. As such, the sample size of 18 in the main experiment appears to be ample in order to detect this difference and increasing the sample size above this value does not add more to the certainty.

\includegraphics{Sup_mat_files/figure-latex/make plots-1.pdf}

\hypertarget{hoop-sizes}{%
\subsection{Hoop Sizes}\label{hoop-sizes}}

Resampling for this experiment took the form of fitting a beta distribution to the \emph{Throwing Experiment} data from Clarke and Hunt (2016) using the fitdistrplus package. The values estimated using this procedure could were then adjusted to look at how reliably different sizes of effects could be detected. We altered the mean value but kept the variance the same to simulate participants shifting in one direction or the other. As can be seen in the figure below, we tested 4 different values (0.05, 0.1, 0.015, and 0.2) which were indicative of participants standing \(X\)\% closer to one side (in this case, in the direction of the smaller hoop). These ditributions were sampled from 5000 times by simulating \(N\) participants (from 3 to 24) and simulating 72 trials. The plot below the distributions shows the uncertainty surrounding the mean estimate for the smallest difference tested (5\%). After 15 participants, the uncertainty surrounding the estimate appears to plateau which demonstrates that the sample size of 21 was sufficient to detect the effect.
\includegraphics{Sup_mat_files/figure-latex/load data and draw distributions-1.pdf}

\hypertarget{two-throws-and-reward}{%
\subsection{Two Throws and Reward}\label{two-throws-and-reward}}

As the hypothesis for both of these experiments was that out intervention would push participants towards being more optimal, we can use the same datasets to look at how uncertainty around the difference would change with varying sample sizes. For these experiments, comparison data was drawn from two unpublished studies; one in which the standard behaviour was observed, and one in which participants were closer to optimal in their performance.

The first dataset is comprised of 40 participants who took part in a version of the Clarke and Hunt (2016) \emph{Throwing Task}. The second dataset is comprised of 60 participants who took part in a computerised version of this task. To compare these datasets, placement positions and standing positions were put on the same scale (0 being the centre and 1 being stood/placed next to the target). Only data for the easiest (smallest separataion of targets) and hard (furthest separation of targets) conditions were considered as these points offered more comparable base performance levels.

\includegraphics{Sup_mat_files/figure-latex/load in data and draw distributions-1.pdf}

\hypertarget{analysis}{%
\section{Analysis}\label{analysis}}

\hypertarget{probability-1}{%
\subsection{Probability}\label{probability-1}}

\includegraphics{Sup_mat_files/figure-latex/Probability show plots-1.pdf}

The results of this model suggest that our participants were sensitive to the probability information (see above figure). In the Biased condition, the average participant fixated the most likely target 68.1\% of the time (95\% HDPI of \textbar{}36.3\%, 94.6\%\textbar{}), compared to 39.5\% (95\% HPDI of \textbar{}19.2\%,59.4\%\textbar{}) in the symmetric condition. The width of these intervals reflects a high degree of uncertainty in fixed effects, due to the range of behaviours exhibited by participants. None-the-less, the HPDI on the difference between these two conditions, \textbar{}3.4, 52.8\textbar{} is largely positive and we can be reasonably confident (P(difference \textgreater{} 0 \textbar{} data ) = 97.5) that the most-likely target is fixated more frequently in the biased condition. The distance between the square targets did not appear to have any consistent effect in the Symmetric condition, however there was a small decrease in fixations towards the ``most likely'' box when the boxes were far apart in the bias condition (dropping from 72.2\%, 95\% HPDI of \textbar{}42.1\%,94.8\%\textbar{} to 64\%, 95\% HPDI of \textbar{}0.302\%,91.4\%\textbar{}). As such, the difference between the conditions were much more pronounced in the close condition (P(Bias \textgreater{} Symmetric\textbar{}data) = 99\%) than in the far condition (P(Bias \textgreater{} Symmetric\textbar{}data) = 96\%). The width of these intervals reflects the range of performance that was exhibited by participants. However, this does show that participants generally made us of this probability information in order to decide where to fixate.

\includegraphics{Sup_mat_files/figure-latex/plots side fixations-1.pdf}

We can now ask whether participants were more likely to make use of the optimal strategy when in the biased condition. We will use the same predictors as above, but this time, to predict the probability of participants fixating either of side boxes over the central box. A random observer, equally likely to look at any of the three boxes, would have a 66\% chance of fixating a Side box. The results suggest that in the Biased condition, participants were more likely to fixate the side boxes than the central box in both the close (94.4\%, HDPI of \textbar{}80.7\%, 99.9\%\textbar{}) and far (95.1\%, HDPI of \textbar{}80.1\%, 100\%\textbar{}) conditions. This contrasts with than when the participants were in the Symmetric condition (82.5\%, HDPI of \textbar{}54.7\%, 99\%\textbar{}) for the close, and , far; 86.3\%, HDPI of \textbar{}51.7\%, 99.9\%\textbar{}) for the far apart targets in the symmetric condition. These results would suggest that the bias present in the Biased condition encouraged participants to fixate the side boxes more often. However, it does not suggest that adding this bias facilitated the use of the optimal strategy as participants did not appear to account for distance in an optimal manner.

\hypertarget{hoop-sizes-1}{%
\subsection{Hoop Sizes}\label{hoop-sizes-1}}

\includegraphics{Sup_mat_files/figure-latex/Sort out plots-1.pdf}

The data from this experiment were analysed using a Bayesian beta regression. The recorded data for standing positions were transformed to be between 0 and 1, with 0 representing the larger hoop and 1 representing the smaller hoop. Therefore, the central point would be 0.,5, and meaning anything above this value would demonstrate a shift in participant's bhaviour away from the mid-point and towards the small hoop. The model included normalised hoop delta as a predictor to see how participants changed position with an increasing distance. This was also entered as a random effect by participant.

The model results confirmed that participants in general had a bias towards standing closer to the small hoop (mean of 0.55, 95\% HDPI of \textbar{}0.508, 0.592\textbar{}). We can be reasonably confident about this results as the p(x \textgreater{} 0.5\textbar{}data) = 98.9\%. This can be seen in the posterior in the above figure. Also, not that distance did not appear to have an effect on position (i.e., participants were generally biased slightly towards the smaller hoop).

\hypertarget{two-throws}{%
\subsection{Two Throws}\label{two-throws}}

\includegraphics{Sup_mat_files/figure-latex/plotting for two throws-1.pdf}

A Bayesian Beta regression was carried out to investigate whether participants performed the task in a more optimal way when they were given the chance to attempt to throw at both targets. The predictors were the No.~of throws and Delta (Hoop separation). The predicted value was the normalised standing position with 0 being central and 1 being next to one of the hoops.
The analysis suggested that there was an overall greater tendency for participants in the One-throw condition to stand further from the centre (mean of 0.199, 95\% HPDI of \textbar{}0.106 , 0.295\textbar{}) than when they were in the Two-throw condition (mean of 0.132, 95\% HPDI of \textbar{}0.063 , 0.214\textbar{}) with P(One-throw \textgreater{} Two-throw\textbar{}data) = 90.1\%.

As can be seen in the above figure, this effect was the strongest for the closest separation which reflects the larger amount of variation in standing position with distance in the Two-throw condition. In general, when in the One-throw condition, participants stood further from the centre (mean of 0.183, 95\% HPDI of \textbar{}0.108 , 0.262\textbar{}) than in the the Two-throw condition (mean of 0.183, 95\% HPDI of \textbar{}0.108 , 0.262\textbar{}) with P(One-throw \textgreater{} Two\_throw\textbar{}data) = 98.1\%.

\hypertarget{reward}{%
\subsection{Reward}\label{reward}}

\includegraphics{Sup_mat_files/figure-latex/make some plots-1.pdf}

\end{document}
